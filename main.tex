\documentclass{article}
\usepackage[utf8]{inputenc}
\usepackage{graphicx}
\usepackage[T1]{fontenc}

\graphicspath{{images/}}

\title{Spadek Swobodny}
\author{Łukasz Suduł}
\date{02.12.2022}

\begin{document}

\maketitle

\section{Wprowadzenie teoretyczne}
    Spadek swobodny jest to fizyczne pojęcie, które w uproszczony sposób opisuje ruch ciała, które na początku znajdowało się w spoczynku i w reakcji na siłę grawitacji zaczęło się poruszać (spadać).
\section{Opis eksperymentu}
    Wyprowadzimy powyższe zależności. Wyprowadzenie to będzie świetnym przykładem zastosowania równania ruchu jednostajnie przyspieszonego. Wprowadzimy układ odniesienia związany z podłożem i zrobimy szkic z oznaczeniami.
    
\includegraphics[scale=0.5]{images/rys0010.jpg}
\caption{Obraz (1)}
\centering

\raggedright Wzór na wysokość ciała w spadku swobodnym to: 
\begin{equation}
h(t)=h_{0}-\frac{gt^2}{2}
\end{equation}
\raggedright
    Jak widać na rys. nr (1), prędkość początkowa jest równa zeru. Przyspieszenie skierowane jest przeciwnie do osi układu odniesienia, zapisujemy je więc ze znakiem minus. Ruch odbywa się tylko wzdłuż jednej osi układu odniesienia, możemy więc posługiwać się skalarami (jedną współrzędna wektorów). Nasze równanie ruchu v(t) przyjmuje postać:
\begin{equation}
v(t)=\sqrt{2hg}
\end{equation}
\section{Wyniki pomiarów}
\begin{table}[h]
\centering
\begin{tabular}{|l|l|}
\hline
Lp. & ŝ     \\ \hline
1   & 3,41  \\ \hline
2   & 16,33 \\ \hline
3   & 26,16 \\ \hline
4   & 22,63 \\ \hline
5   & 17,62 \\ \hline
6   & 16,10 \\ \hline
7   & 23,06 \\ \hline
8   & 4,80  \\ \hline
9   & 12,79 \\ \hline
10  & 15,22 \\ \hline
11  & 25,40 \\ \hline
12  & 33,83 \\ \hline
13  & 32,04 \\ \hline
14  & 18,06 \\ \hline
15  & 29,89 \\ \hline
16  & 46,41 \\ \hline
17  & 6,58  \\ \hline
18  & 29,65 \\ \hline
19  & 27,02 \\ \hline
20  & 49,65 \\ \hline
\end{tabular}
\caption{Wyniki Pomiarów}
\end{table}

\includegraphics[scale=0.45]{images/wykres.png}
\caption{Wykres zależności drogi od czasu}
\centering

\raggedright
\section{Wnioski}
W spadku swobodnym ciało w początkowym położeniu posiada prędkość wynoszącą 0, za to w położeniu ostatecznym, wynoszącym wysokość 0 posiada prędkość $$\sqrt{2hg}$$
    
\end{document}
